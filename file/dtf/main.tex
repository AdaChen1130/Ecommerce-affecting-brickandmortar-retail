\documentclass{article}

% Language setting
% Replace `english' with e.g. `spanish' to change the document language
\usepackage[english]{babel}

% Set page size and margins
% Replace `letterpaper' with `a4paper' for UK/EU standard size
\usepackage[letterpaper,top=2cm,bottom=2cm,left=3cm,right=3cm,marginparwidth=1.75cm]{geometry}

% Useful packages
\usepackage{amsmath}
\usepackage{graphicx}
\usepackage[colorlinks=true, allcolors=blue]{hyperref}

\title{Transforming in the Digital Tide: An In-depth Analysis of the Impact of E-Commerce on Brick-and-Mortar Retail in China, 2018-2021}
\author{You}

\begin{document}
\maketitle

\begin{abstract}
Your abstract.
\end{abstract}

\section{Introduction}
The rapid expansion and integration of e-commerce in China has significantly changed the retail landscape in recent years. This report presents an in-depth analytical exploration of this transformation, focusing on how the burgeoning e-commerce sector influences and reshapes traditional brick-and-mortar retail. The comprehension of global retail trends necessitates a thorough awareness of the dynamics in China, which currently has a prominent position in Internet commerce.Our analysis begins with a comprehensive overview of the retail market in China, laying the groundwork to appreciate the scale and pace of change. We delve into the historical data on retail sales, examining online and offline channels to identify trends, patterns, and significant shifts in consumer behaviour. The aforementioned historical framework serves as a foundation upon which the contemporary emergence of e-commerce can be situated.We then turn our attention to the core of this report: the interplay between e-commerce and traditional retail. Here, we dissect the growth rates of online sales against their offline counterparts, revealing the impact of digital platforms on market share and consumer preferences and how traditional retailers adapt to this new era. This analysis seeks to illuminate the nuances of the transition from physical stores to digital platforms, highlighting areas of synergy, competition, and transformation.While our investigation touches on topics pertinent to various stakeholders in the retail industry, it is essential to clarify that this report's primary objective is analytical. We aim to offer a detailed, data-driven examination of the evolving retail landscape in China. Our focus is on unpacking the complex dynamics at play and providing a narrative that is rich in detail and insights.

In conclusion, this report seeks to contribute to the broader dialogue on the future of retail in the digital age. Through a comprehensive analysis of the Chinese market, recognised for its pioneering advancements in e-commerce and digital consumer interaction, our objective is to elucidate discernible patterns and trends that can shape and impact the worldwide retail industry.


\section{Literature Review}
The e-commerce landscape has been transformative since Amazon's inception in 1994. Beginning as a bookseller, Amazon expanded into a diverse global marketplace, achieving sales of \$136 billion by 2016. In parallel, the founding of Alibaba.com in China in 1999 marked the beginning of a significant shift in the Asian e-commerce market. Alibaba's 2018 Singles' Day sale, amassing \$30.8 billion in transactions, underscored the global impact of e-commerce, dwarfing the sales of Black Friday and Cyber Monday in the United States\cite{luo2019commerce}

This evolution of e-commerce as a novel business model has profoundly affected consumer behaviours and consumption patterns\cite{fan2018alibaba}. In developed countries, the discourse around e-commerce has focused on its welfare implications—ranging from reduced consumer search costs and increased product variety to its effects on traditional retail sectors, such as the decline seen in physical bookstores.

The growth trajectory of e-commerce in China is particularly noteworthy in this global narrative. By 2017, China boasted 772 million internet users, with 533 million participating in online shopping. The annual e-commerce trade volume soared from RMB 930 billion in 2004 to RMB 29,160 billion in 2017, highlighting a compound annual growth rate of 30\%. This explosive growth was mirrored in online retail sales, which increased from RMB 125.7 billion in 2008 to RMB 5,155.6 billion in 2016. Internationally, China's contribution to global e-commerce transactions rocketed from less than 1\% a decade ago to over 40\%, surpassing the combined totals of economic powerhouses like France, Germany, Japan, the UK, and the US\cite{luo2019commerce}.

Yet, within China, the development of e-commerce reveals significant regional and urban-rural disparities. Central provinces such as Beijing, Shanghai, Zhejiang, and Guangdong recorded high percentages of online retail sales in 2015, while in contrast, less than 2\% of retail sales were online in nine inland provinces. This dichotomy extends to the urban-rural divide, with 72.6\% of the national total urban internet users compared to 27.4\% in rural areas\cite{mofcom2016ecommerce}. However, rural online retail transactions have grown more rapidly, increasing from RMB 353 billion in 2015 to RMB 895 billion in 2016\cite{mofcom2016ecommerce}.

This backdrop sets the stage for our investigation into how the e-commerce surge impacts traditional brick-and-mortar retail dynamics in China. As we delve into the nuances of this relationship, we aim to unravel the complexities of a market in transition, where the traditional and the digital coexist and compete, shaping the future of retail in one of the world's most vibrant economies.
\section{Assumption}

\begin{itemize}
    \item \textbf{Data Reliability:} Assuming the data for e-commerce and traditional retail in China is accurate and representative of market trends.
    \item \textbf{Consistent Consumer Behavior:} Assuming consumer preferences remained stable, focusing solely on the impact of e-commerce on sales dynamics.
    \item \textbf{Broad Representation:} Assuming the data encompasses diverse geographic and demographic segments across China.
    \item \textbf{Technology Access:} Assuming widespread access to and usage of e-commerce platforms across different consumer segments.
    \item \textbf{Clear Retail Segmentation:} Assuming a distinct separation between online and offline retail channels for analysis purposes.
\end{itemize}

\section{Prediction}
\begin{itemize}
    \item The increasing negative correlation between online and offline retail channels in China signals a consumer shift towards digital platforms, impacting traditional brick-and-mortar store traffic.
    \item The COVID-19 pandemic has accelerated the move to e-commerce, likely causing enduring changes in consumer behavior and underscoring the need for retail agility and digital readiness.
\end{itemize}

\section{Data Section}

\subsection*{Data Sources and Collection}

\subsubsection*{Sources}
Our primary data source is the official website of the National Bureau of Statistics of China, which provides monthly growth figures for online and offline consumption. This rich dataset offers an in-depth view of the evolving retail landscape in China from 2018 to 2021, encompassing various product categories and retail formats.

\subsubsection*{Data Collection}
The collection process involved the use of web scraping techniques to efficiently gather large volumes of data. By deploying Python-based scraping tools, we were able to automate the extraction of relevant data, ensuring accuracy and timeliness.

\subsection*{Data Description}

\subsubsection*{Overview}
The dataset spans from 2018 to 2021, detailing monthly consumption amounts for both online and offline channels. It includes relative growth rates, breaking down offline sales into different categories to reveal trends and shifts in consumer behavior.

\subsubsection*{Key Insights}
\begin{itemize}
    \item The data reveals a dynamic interplay between online and offline retail channels, with noticeable trends and shifts over the examined period.
    \item Offline sales are categorized into various product types, offering insights into consumer preferences and market dynamics.
\end{itemize}

\subsection*{Data Processing}

\subsubsection*{Approach}
In processing the data, our aim was to create a cohesive and comprehensive dataset that accurately reflects market dynamics. This involved:
\begin{enumerate}
    \item Data Integration and Summarization: Combining various datasets to form a unified view of the retail landscape.
    \item Handling Data Shortages: For January of each year, where data is typically scarce, we applied an averaging technique. The average of February's data was used to estimate January's figures, with February data remaining unchanged in cumulative metrics.
    \item Correlation Analysis: Investigating the negative correlation between online and offline sales, particularly under the influence of COVID-19, and its impact on the overall retail sales.
\end{enumerate}

\subsection*{Data Crawling with Python}


\subsubsection*{Methodology}
Our data collection from the National Bureau of Statistics of China was executed through a Python-based web scraping approach. Key steps of the methodology included:

\begin{enumerate}
    \item \textbf{Data Extraction}: Crafting HTTP POST requests with appropriate headers and parameters to access the targeted datasets. The data, spanning from 2019 to 2023, included detailed retail sales figures and growth rates.
    \item \textbf{Parsing and Cleaning}: Employing custom functions to parse JSON responses, extract relevant data points, and handle missing or incomplete data for consistency and accuracy.
    \item \textbf{Data Structuring}: Converting the scraped data into a structured format using pandas DataFrames, facilitating easy manipulation and analysis in subsequent stages.
\end{enumerate}
This streamlined and ethical web scraping process ensured the collection of comprehensive and reliable data, forming the backbone of our analysis.

\subsection*{Key Considerations in Data Handling}

\begin{enumerate}
    \item \textbf{Integration of Diverse Data Sources:} The challenge was to merge data from different formats and sources into a single, coherent dataset.
    \item \textbf{Handling Seasonal Shortages:} The approach to address data scarcity in January each year was crucial to maintain the integrity of the analysis.
    \item \textbf{Cumulative Data Considerations:} In dealing with cumulative data, special attention was paid to ensure that modifications (like the averaging for January) did not skew the overall trends.
    \item \textbf{Online-Offline Correlation and COVID-19 Impact:} It was essential to understand and articulate how the pandemic influenced retail dynamics, affecting both online and offline channels.
\end{enumerate}

\subsection*{Conclusion}

This data-centric approach lays the foundation for a comprehensive analysis of the evolving retail landscape in China, highlighting key trends and patterns that have implications for the global retail sector.




\subsection{How to add Comments and Track Changes}

Comments can be added to your project by highlighting some text and clicking ``Add comment'' in the top right of the editor pane. To view existing comments, click on the Review menu in the toolbar above. To reply to a comment, click on the Reply button in the lower right corner of the comment. You can close the Review pane by clicking its name on the toolbar when you're done reviewing for the time being.

Track changes are available on all our \href{https://www.overleaf.com/user/subscription/plans}{premium plans}, and can be toggled on or off using the option at the top of the Review pane. Track changes allow you to keep track of every change made to the document, along with the person making the change. 

\subsection{How to add Lists}

You can make lists with automatic numbering \dots

\begin{enumerate}
\item Like this,
\item and like this.
\end{enumerate}
\dots or bullet points \dots
\begin{itemize}
\item Like this,
\item and like this.
\end{itemize}

\subsection{How to write Mathematics}

\LaTeX{} is great at typesetting mathematics. Let $X_1, X_2, \ldots, X_n$ be a sequence of independent and identically distributed random variables with $\text{E}[X_i] = \mu$ and $\text{Var}[X_i] = \sigma^2 < \infty$, and let
\[S_n = \frac{X_1 + X_2 + \cdots + X_n}{n}
      = \frac{1}{n}\sum_{i}^{n} X_i\]
denote their mean. Then as $n$ approaches infinity, the random variables $\sqrt{n}(S_n - \mu)$ converge in distribution to a normal $\mathcal{N}(0, \sigma^2)$.


\subsection{How to change the margins and paper size}

Usually the template you're using will have the page margins and paper size set correctly for that use-case. For example, if you're using a journal article template provided by the journal publisher, that template will be formatted according to their requirements. In these cases, it's best not to alter the margins directly.

If however you're using a more general template, such as this one, and would like to alter the margins, a common way to do so is via the geometry package. You can find the geometry package loaded in the preamble at the top of this example file, and if you'd like to learn more about how to adjust the settings, please visit this help article on \href{https://www.overleaf.com/learn/latex/page_size_and_margins}{page size and margins}.

\subsection{How to change the document language and spell check settings}

Overleaf supports many different languages, including multiple different languages within one document. 

To configure the document language, simply edit the option provided to the babel package in the preamble at the top of this example project. To learn more about the different options, please visit this help article on \href{https://www.overleaf.com/learn/latex/International_language_support}{international language support}.

To change the spell check language, simply open the Overleaf menu at the top left of the editor window, scroll down to the spell check setting, and adjust accordingly.

\subsection{How to add Citations and a References List}

You can simply upload a \verb|.bib| file containing your BibTeX entries, created with a tool such as JabRef. You can then cite entries from it, like this: \cite{greenwade93}. Just remember to specify a bibliography style, as well as the filename of the \verb|.bib|. You can find a \href{https://www.overleaf.com/help/97-how-to-include-a-bibliography-using-bibtex}{video tutorial here} to learn more about BibTeX.

If you have an \href{https://www.overleaf.com/user/subscription/plans}{upgraded account}, you can also import your Mendeley or Zotero library directly as a \verb|.bib| file, via the upload menu in the file-tree.

\subsection{Good luck!}

We hope you find \cite{greenwade93}Overleaf useful, and do take a look at our \href{https://www.overleaf.com/learn}{help library} for more tutorials and user guides! Please also let us know if you have any feedback using the Contact U\cite{luo2019commerce} link at the bottom of the Overleaf menu --- or use the contact form at \url{https://www.overleaf.com/contact}.

\bibliographystyle{alpha}
\bibliography{sample}

\end{document}


@article{luo2019commerce,
  title={E-Commerce development and household consumption growth in China},
  author={Luo, Xubei and Wang, Yue and Zhang, Xiaobo},
  journal={World Bank Policy Research Working Paper},
  number={8810},
  year={2019}
}
@article{luo2019commerce,
  title={E-Commerce development and household consumption growth in China},
  author={Luo, Xubei and Wang, Yue and Zhang, Xiaobo},
  journal={World Bank Policy Research Working Paper},
  number={8810},
  year={2019}
}